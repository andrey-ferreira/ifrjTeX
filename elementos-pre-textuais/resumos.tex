%
% Documento: Resumos (Português e inglês)
%

\begin{resumo}
Síntese do trabalho em texto cursivo contendo um único parágrafo. O resumo é a apresentação clara, concisa e seletiva do trabalho.
No resumo deve-se incluir, preferencialmente, nesta ordem: brevíssima introdução ao assunto do trabalho de pesquisa (qualificando-o quanto à sua natureza), o que será feito no trabalho (objetivos), como ele será desenvolvido (metodologia), quais serão os principais resultados e conclusões esperadas, bem como qual será o seu valor no contexto acadêmico. Este resumo não deve ultrapassar 500 palavras.

\textbf{Palavras-chave}: latex. abntex. modelo.
(Entre 3 a 6 palavras ou termos, separados por ponto, descritores do trabalho. As palavras-chaves são Utilizadas para indexação.
\end{resumo}


% Resumo (Inglês)

\begin{resumo}[Abstract]
Translation of the abstract into english, possibly adapting or slightly changing the text in order to adjust it to the grammar of english educated.

\textbf{Keywords}: latex. abntex. template.
\end{resumo}