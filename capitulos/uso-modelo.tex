\chapter{Uso do modelo}

Por se tratar de uma extensão da classe \verb|abntex2|, o modelo herda todas as opções, ambientes e comandos definidos tanto na própria classe \verb|abntex2| \autocite{abntex2classe} quanto na classe \verb|memoir| \autocite{madsen2021memoir}.  
Dessa forma, o usuário pode utilizar normalmente todos os recursos já disponíveis nessas classes.

\section{Opções da classe \texttt{ifrjtex}}

A classe \texttt{ifrjtex} oferece as opções \verb|artigo| e \verb|projeto|, que ajustam automaticamente a folha de rosto conforme o formato exigido pelo \textcite{ifrjtccs}.  

Caso o usuário prefira editar manualmente esses elementos, recomenda-se redefinir as macros específicas responsáveis por sua impressão, fornecidas pela classe \verb|abntex2|.

\subsection{Projetos de Pesquisa}

De acordo com o Manual de Apresentação de Trabalhos Acadêmicos do IFRJ:

\begin{citacao}
O projeto de pesquisa é um planejamento das etapas do trabalho que será
desenvolvido. Sua estrutura pode variar de acordo com as normas estabelecidas pela
instituição, sendo constituído pela parte externa e pela parte interna. A externa
constitui-se de capa (obrigatória), e a interna é organizada por elementos pré-textuais, elementos textuais e elementos pós-textuais.  \autocite[p. 82]{ifrjtccs}
\end{citacao}

Para gerar automaticamente os elementos pré-textuais definidos no Manual, basta carregar a classe com a opção \verb|projeto|:

\begin{verbatim}
\documentclass[projeto]{ifrjtex}
\end{verbatim}

Além dos elementos obrigatórios, a classe oferece a macro adicional \verb|\areaconcentracao{}|, voltada a cursos que exigem a indicação de área de concentração ou linha de pesquisa.  
Quando preenchida, essa informação é impressa na folha de rosto, logo após o preâmbulo, com o rótulo padrão “\textbf{Área de concentração:}”.

O rótulo pode ser alterado facilmente. Por exemplo:

\begin{verbatim}
\areaconcentracao[Linha de pesquisa: ]{Nome da linha}
\end{verbatim}

Nesse caso, o texto impresso será “\textbf{Linha de pesquisa:}” em vez de “\textbf{Área de concentração:}”.

\subsection{Artigos}

Segundo o Manual de Apresentação de Trabalhos Acadêmicos:

\begin{citacao}
Os autores de artigos submetidos a periódicos externos e de acesso livre, que utilizaram esse tipo de publicação como TCC, deverão comunicar à Biblioteca do campus sobre a existência da pesquisa, enviando uma cópia do artigo para depósito no Repositório Institucional (RI) do IFRJ. É obrigatória a inclusão dos seguintes elementos pré-textuais:  
a) capa, conforme o modelo do Apêndice A e seção 2.1;  
b) folha de rosto, conforme o Apêndice E;  
c) verso da folha de rosto, com ficha catalográfica elaborada por bibliotecário da Instituição;  
d) folha de aprovação, contendo as assinaturas da banca, que deve ser digitalizada e inserida no documento digital. \autocite[p. 102]{ifrjtccs}
\end{citacao}

Para gerar automaticamente esses elementos, utilize a opção \verb|artigo| ao carregar a classe:

\begin{verbatim}
\documentclass[artigo]{ifrjtex}
\end{verbatim}

Para inserir o artigo publicado, recomenda-se o uso do pacote \verb|pdfpages| \autocite{pdfpages}, conforme o exemplo:

\begin{verbatim}
\includepdf[pages=-]{meuartigo.pdf}
\end{verbatim}

\section{Referências e citações}

Esta classe utiliza o pacote \verb|biblatex| com o estilo \verb|abnt|, em substituição aos estilos tradicionais do \verb|bibtex|.  
Conforme indicado por \textcite{araujopacote}:

\begin{citacao}
Sendo totalmente implementado em \LaTeX, o \verb|biblatex-abnt| substitui os arquivos de estilo \verb|bibtex| (\texttt{abntex2-alf.bst}, \texttt{abntex2-num.bst}) e o pacote \texttt{abntex2cite.sty}. Assim, deve-se utilizar exclusivamente o \verb|biblatex-abnt| e as macros padrão do Bib\LaTeX. \autocite[p. 3]{araujopacote}
\end{citacao}

É recomendável consultar a documentação completa do pacote em\\ \url{https://www.ctan.org/pkg/biblatex-abnt}.  
De acordo com \textcite{marquesbiblatex}, o \verb|biblatex-abnt 3.4| requer as versões \verb|biblatex 3.8| e \verb|biber 2.8|.  
Caso ocorram erros de compilação, verifique se os pacotes estão atualizados.

Exemplos de entradas compatíveis com o arquivo \verb|refbase.bib| podem ser encontrados em \textcite{marquesbiblatex}.  
A lista completa de tipos de entrada e campos disponíveis está em \citetitle{lehman2006biblatex}.  
Para organização das referências, recomenda-se o uso de gerenciadores como \textcite{Zotero} e \textcite{JabRef2009}.

\subsection{Citações livres}\label{citacoesLivres}

As citações livres reproduzem as ideias de um autor sem transcrever literalmente seu texto.  
Servem para embasar, explicar ou relacionar argumentos do trabalho.  
Todas as obras citadas devem obrigatoriamente constar nas referências, garantindo o respeito aos direitos autorais.

\textcite{maturana:2003} defende um princípio de lógica segundo o qual “quando dizemos \ldots”.  
Já \textcite{teste:2004} argumentam que \ldots\mbox{ }O uso do termo \textit{et al.} é automático quando há mais de três autores, mas pode ser desativado com a opção \verb|abnt-no-etal-label|.

Também é possível inserir a referência entre parênteses, conforme o exemplo:

\begin{quote}
(\textsc{AUTOR}, ano)
\end{quote}

\subsection{Citações literais}\label{citacoesLiterais}

As citações literais reproduzem fielmente as palavras do autor.  
Citações com mais de três linhas devem ser destacadas em parágrafo recuado (4 cm da margem esquerda), com fonte menor e espaçamento simples:

\begin{citacao}
Desse modo, opera-se uma ruptura decisiva entre a reflexividade filosófica, isto é, a possibilidade do sujeito de pensar e refletir, e a objetividade científica. Encontramo-nos num ponto em que o conhecimento científico está sem consciência. Sem consciência moral, sem consciência reflexiva e também subjetiva. Cada vez mais, o desenvolvimento extraordinário do conhecimento científico vai tornar menos praticável a própria possibilidade de reflexão do sujeito sobre a sua pesquisa. \autocite[p.28]{morinmoigne:2000}
\end{citacao}

O ambiente \verb|citacao| gera automaticamente esse formato:

\begin{lstlisting}
\begin{citacao}
    Texto da citação literal.
\end{citacao}
\end{lstlisting}


Citações com menos de três linhas permanecem no corpo do texto, entre aspas, seguidas do autor, ano e página:

\begin{quote}
``Assumo que não posso fazer referência a entidades independentes de mim para construir meu explicar'' \autocite[p.~35]{maturana:2003}.
\end{quote}

\subsection{Comandos úteis para citações}\label{referenciasUtilizadas}

Alguns exemplos de comandos e seus resultados:

\begin{itemize}
    \item \verb|\textcite{memoir}| → \textcite{memoir}
    \item \verb|\autocite{Goossens2007}| → \autocite{Goossens2007}
    \item \verb|\textcites{acmsurveys,carvalho:2001}| → \textcites{acmsurveys,carvalho:2001}
    \item \verb|\autocites{acmsurveys,Buerger1989}| → \autocites{acmsurveys,Buerger1989}
    \item \verb|\textapud[p.~10]{maturana:2003}[p.~20]{morinmoigne:2000}| → \textapud[p.~10]{maturana:2003}[p.~20]{morinmoigne:2000}
    \item \verb|\apud{maturana:2003}{morinmoigne:2000}| → \apud{maturana:2003}{morinmoigne:2000}
    \item \verb|\citeauthor{abntex2classe}| → \citeauthor{abntex2classe}
    \item \verb|\citeyear{abntex2classe}| → \citeyear{abntex2classe}
    \item \verb|\fullcite{Goossens2007}| → \fullcite{Goossens2007}
\end{itemize}


\chapter{Elementos flutuantes e referências cruzadas}
\label{chap:ef}

Este capítulo apresenta exemplos de inclusão de figuras, tabelas, equações e outros elementos, com indexação automática em suas respectivas listas.  
A numeração de figuras, tabelas e equações é gerada automaticamente.  
As referências cruzadas utilizam os comandos \verb|\label{}| e \verb|\ref{}|, o que elimina a necessidade de ajustar manualmente a numeração ao mover, adicionar ou remover elementos no texto.

\section{Equações}
\label{sec:equacoes}

A \autoref{eq:laplace} mostra a transformada de Laplace, enquanto a \autoref{eq:dft} apresenta a formulação da transformada discreta de Fourier bidimensional\footnote{Observe a formatação esteticamente precisa destas equações.}.

\begin{lstlisting}[language=TeX, breaklines=true, basicstyle=\ttfamily\small,  label={lst:equacoes}]
\begin{equation}
    X(s) = \int\limits_{t = -\infty}^{\infty} x(t) \, \text{e}^{-st} \, dt
    \label{eq:laplace}
\end{equation}

\begin{equation}
    F(u,v) = \sum_{m=0}^{M-1} \sum_{n=0}^{N-1} 
    f(m,n)\exp\left[-j2\pi\left(\frac{um}{M}+\frac{vn}{N}\right)\right]
    \label{eq:dft}
\end{equation}
\end{lstlisting}

\begin{equation}
    X(s) = \int\limits_{t = -\infty}^{\infty} x(t) \, e^{-st} \, dt
    \label{eq:laplace}
\end{equation}

\begin{equation}
    F(u, v) = \sum_{m = 0}^{M - 1} \sum_{n = 0}^{N - 1} f(m, n)
    \exp \left[ -j 2 \pi \left( \frac{u m}{M} + \frac{v n}{N} \right) \right]
    \label{eq:dft}
\end{equation}

\section{Elementos flutuantes}

Tabelas, quadros, gráficos e figuras são amplamente utilizados em trabalhos acadêmicos, cada qual com especificações próprias definidas pela ABNT \autocite{NBR14724:2024}.  
Todos devem ser devidamente referenciados no texto.

O Quadro \ref{quadroex1} apresenta um resumo das principais diferenças entre esses elementos.

\begin{quadro}[H]
\small
\centering
\caption{Comparativo entre tipos de elementos flutuantes}
\label{quadroex1}
\begin{tabular}{|l|m{4cm}|m{4cm}|m{4cm}|}
\hline
 & \textbf{Tabela} & \textbf{Quadro} & \textbf{Figura} \\ \hline
\textbf{Formato} & Bordas laterais abertas. & Bordas totalmente fechadas. & Fotos, mapas, gráficos, gravuras etc. \\ \hline
\textbf{Uso} & Dados quantitativos. & Dados qualitativos. & Ilustração de dados e informações. \\ \hline
\textbf{Elementos} & Título, cabeçalho, conteúdo e fonte. & Título, fonte e notas. & Título, numeração e fonte. \\ \hline
\textbf{Divisão} & Linhas horizontais. & Linhas horizontais e verticais. & - \\ \hline
\textbf{Formatação} & Título e número acima; fonte abaixo. & Título e número acima; fonte abaixo. & Título acima; fonte abaixo. \\ \hline
\end{tabular}
\fonte{16cm}{Elaborado pelo autor.}
\end{quadro}

A inclusão de um elemento flutuante segue a estrutura abaixo:

\begin{lstlisting}[language=TeX, breaklines=true, basicstyle=\ttfamily\small,  label={lst:estrutura}]
\begin{nome do ambiente}[H]
    \centering
    \caption{Legenda do elemento}
    \label{etiqueta}
    inserção do conteúdo
    \fonte{tamanho do elemento}{fonte do elemento}
\end{nome do ambiente}
\end{lstlisting}

\begin{quadro}[H]
\small
\centering
\caption{Campos utilizados em elementos flutuantes}
\label{quadroex2}
\begin{tabular}{|m{4cm}|m{8cm}|}
\hline
\verb|nome do ambiente| & \verb|figure| (figuras), \verb|grafico| (gráficos), \verb|table| (tabelas), \verb|quadro| (quadros). \\ \hline
\verb|legenda| & Texto descritivo do elemento, inserido em \verb|\caption{}|. \\ \hline
\verb|etiqueta| & Nome usado para referenciar o elemento via \verb|\ref{}|. \\ \hline
\verb|fonte do elemento| & Indicação da origem do elemento, inserida em \verb|\fonte{}{}|. \\ \hline
\end{tabular}
\fonte{13cm}{Elaborado pelo autor.}
\end{quadro}

\subsection{Figuras e gráficos}

As figuras e gráficos servem para ilustrar dados ou conceitos, podendo ser compostos por fotos, mapas, diagramas ou representações gráficas.  
A Figura \ref{fig:logo} aparece automaticamente na lista de figuras.

\begin{figure}[H]
\centering
\caption{Logo do projeto \LaTeX.}
\includegraphics[width=13cm]{figuras/LaTeXproject.png}
\fonte{13cm}{\textcite{LaTeX2009}}
\label{fig:logo}
\end{figure}

\begin{lstlisting}[language=TeX, breaklines=true, basicstyle=\ttfamily\small, label={lst:figura}]
\begin{figure}[H]
    \centering
    \caption{Logo do projeto \LaTeX.}
    \includegraphics[width=13cm]{figuras/LaTeXproject.png}
    \fonte{13cm}{\textcite{LaTeX2009}}
    \label{fig:logo}
\end{figure}
\end{lstlisting}

O Gráfico \ref{gr:exgrafico} ilustra um exemplo de gráfico inserido com o ambiente \verb|grafico|:

\begin{grafico}[H]
\centering
\caption{Evolução da Lei Orçamentária Anual (LOA) e número de matrículas no IFMG}
\includegraphics[width=0.85\linewidth]{figuras/loa-e-matriculas.png}
\fonte{0.85\linewidth}{\textcite{ifmg2019mobilizacao}}
\label{gr:exgrafico}
\end{grafico}
\pagebreak

\begin{lstlisting}[language=TeX, breaklines=true, basicstyle=\ttfamily\small, label={lst:grafico}]
\begin{grafico}[H]
    \centering
    \caption{Evolução da Lei Orçamentária Anual (LOA) e número de matrículas no IFMG}
    \includegraphics[width=0.85\linewidth]{figuras/loa-e-matriculas.png}
    \fonte{0.85\linewidth}{\textcite{ifmg2019mobilizacao}}
    \label{gr:exgrafico}
\end{grafico}
\end{lstlisting}

\subsection{Tabelas}

As tabelas devem seguir as normas descritas por \textcite{ibge1993}.  
Elas apresentam dados quantitativos, com linhas horizontais e bordas laterais abertas.  
A classe \verb|abntex2| fornece a macro \verb|\IBGEtab{}{}{}|, que formata tabelas conforme o padrão do IBGE.

\begin{table}[H]
\IBGEtab{
\centering
\caption{Pessoas residentes em domicílios particulares, por sexo e situação do domicílio - Brasil, 1980}
\label{tab:residentes}
}{
\begin{tabular}{m{2.5cm}ccc}
\toprule
Situação do domicílio & Total & Mulheres & Homens \\
\midrule
Total & 117.960.301 & 59.595.332 & 58.364.969 \\
Urbana & 79.972.931 & 41.115.439 & 38.857.492 \\
Rural & 37.987.370 & 18.479.893 & 19.507.477 \\
\bottomrule
\end{tabular}
}{
\fonte{10cm}{Dados fictícios.}
}
\end{table}

\begin{lstlisting}[language=TeX, breaklines=true, basicstyle=\ttfamily\small,  label={lst:tabela}]
\begin{table}[H]
\IBGEtab{
    \centering
    \caption{Pessoas residentes em domicílios particulares, por sexo e situação do domicílio - Brasil, 1980}
    \label{tab:residentes}
}{
    \begin{tabular}{m{2.5cm}ccc}
    \toprule
    Situação do domicílio & Total & Mulheres & Homens \\
    \midrule
    Total & 117.960.301 & 59.595.332 & 58.364.969 \\
    Urbana & 79.972.931 & 41.115.439 & 38.857.492 \\
    Rural & 37.987.370 & 18.479.893 & 19.507.477 \\
    \bottomrule
    \end{tabular}
}{
    \fonte{10cm}{Fonte: Dados fictícios.}
}
\end{table}
\end{lstlisting}

\subsection{Quadros}

Os quadros são formados por linhas verticais e horizontais, sendo mais adequados para dados qualitativos.  
Devem ser criados dentro do ambiente \verb|quadro|, podendo utilizar \verb|tabular|, \verb|tabularray| ou outros formatos de tabela.

\begin{lstlisting}[language=TeX, breaklines=true, basicstyle=\ttfamily\small,  label={lst:quadro}]
\begin{quadro}[H]
    \centering
    \caption{Exemplo de quadro simples}
    \label{qua:ex}
    \begin{tabular}{|m{4cm}|m{5cm}|m{3cm}|}
        \hline
        \textbf{Título 1} & \textbf{Título 2} & \textbf{Título 3} \\ \hline
        Texto aqui & Lorem ipsum dolor ... & Ut accumsan sapien eget magna ... \\ \hline
        Texto da segunda linha & bla bla bla & bla bla bla \\ \hline
    \end{tabular}
    \fonte{13cm}{Elaborado pelo autor.}
\end{quadro}
\end{lstlisting}

\begin{quadro}[H]
\centering
\caption{Exemplo de quadro simples}
\label{qua:ex}
\begin{tabular}{|m{4cm}|m{4cm}|m{4cm}|}
\hline
\textbf{Título 1} & \textbf{Título 2} & \textbf{Título 3} \\ \hline
Texto aqui & Lorem ipsum dolor ... & Ut accumsan sapien eget magna ... \\ \hline
Texto da segunda linha & bla bla bla & bla bla bla \\ \hline
\end{tabular}
\fonte{13cm}{Elaborado pelo autor.}
\end{quadro}



\section{Exemplo de criação de cronograma}

Em projetos de pesquisa, é comum a necessidade de apresentar um plano de trabalho que descreva as etapas e o cronograma de execução.  
A seguir, mostra-se um exemplo de criação de cronograma utilizando o pacote \verb|pgfgantt|, com algumas personalizações. 

\newpage
\begin{lstlisting}[language={[LaTeX]TeX}, label={lst:cronograma}, breaklines=true]
% Descrição das etapas
\begin{etapas}
    \item \label{e1} Descrição da etapa 1
    \item \label{e2} Descrição da etapa 2
    \item \label{e3} Sed consequat tellus et tortor. Ut tempor laoreet quam.
    \item \label{e4} Sed consequat tellus et tortor. Ut tempor laoreet quam. 
    \item \label{e5} Sed consequat tellus et tortor. Ut tempor laoreet quam. 
    \item \label{e6} Sed consequat tellus et tortor. Ut tempor laoreet quam. 
    \item \label{e7} Sed consequat tellus et tortor. Ut tempor laoreet quam. 
\end{etapas}

% Diagrama com referências às etapas
\begin{center}
    \begin{ganttchart}[vgrid, hgrid]{1}{24}
        \gantttitle{2021}{12} % 12 meses em 2021
        \gantttitle{2022}{12} \\ % 12 meses em 2022
        \gantttitlelist{1,...,12}{1} % meses 1 ao 12 de 2021
        \gantttitlelist{1,...,12}{1} \\ % meses 1 ao 12 de 2022
        \ganttbar[progress=65]{\ref{e1}}{1}{7} \\
        \ganttbar{\ref{e2}}{7}{12} \\
        \ganttbar{\ref{e3}}{13}{18} \\
        \ganttbar{\ref{e4}}{18}{24}
    \end{ganttchart}
    
    \begin{ganttchart}[vgrid, hgrid]{1}{12}
        \gantttitle{2023}{12} \\ % 12 meses em 2023
        \gantttitlelist{1,...,12}{1} \\ % meses 1 ao 12 de 2023
        \ganttbar{\ref{e5}}{1}{7} \\
        \ganttbar{\ref{e6}}{8}{10} \\
        \ganttbar{\ref{e7}}{10}{12}
    \end{ganttchart}
\end{center}
\end{lstlisting}

\pagebreak
A saída correspondente é apresentada a seguir:

\begin{etapas}
    \item \label{e1} Descrição da etapa 1
    \item \label{e2} Descrição da etapa 2
    \item \label{e3} Sed consequat tellus et tortor. Ut tempor laoreet quam. 
    \item \label{e4} Sed consequat tellus et tortor. Ut tempor laoreet quam. 
    \item \label{e5} Sed consequat tellus et tortor. Ut tempor laoreet quam. 
    \item \label{e6} Sed consequat tellus et tortor. Ut tempor laoreet quam. 
    \item \label{e7} Sed consequat tellus et tortor. Ut tempor laoreet quam. 
\end{etapas}

\begin{center}
\begin{ganttchart}[vgrid,hgrid]{1}{24}
    \gantttitle{2021}{12}
    \gantttitle{2022}{12} \\
    \gantttitlelist{1,...,12}{1}
    \gantttitlelist{1,...,12}{1} \\
    \ganttbar[progress=65]{\ref{e1}}{1}{7} \\
    \ganttbar{\ref{e2}}{7}{12} \\
    \ganttbar{\ref{e3}}{13}{18} \\
    \ganttbar{\ref{e4}}{18}{24}
\end{ganttchart}

\begin{ganttchart}[vgrid,hgrid]{1}{12}
    \gantttitle{2023}{12}\\
    \gantttitlelist{1,...,12}{1} \\
    \ganttbar{\ref{e5}}{1}{7} \\
    \ganttbar{\ref{e6}}{8}{10} \\
    \ganttbar{\ref{e7}}{10}{12}
\end{ganttchart}
\end{center}

