%
% Documento: Conclusão
%

\chapter{Considerações finais}
\label{chap:Considerações finais}

Espera-se que o uso do estilo de formatação em \LaTeX, desenvolvido em conformidade com as Normas para Elaboração de Trabalhos Acadêmicos do IFRJ \autocite{ifrjtccs}, contribua para padronizar a apresentação dos documentos produzidos na instituição e, ao mesmo tempo, facilite o processo de escrita e revisão por parte de seus autores. 

Para usuários iniciantes em \LaTeX, além das referências especializadas já indicadas, há diversos recursos adicionais disponíveis na Internet, como os repositórios \cite{CTAN2009} e comunidades de apoio e documentação \cite{TeX-Br2009, Wikibooks2009}, que podem auxiliar no aprofundamento do aprendizado.

Recomenda-se, ainda, o uso de gerenciadores de referências bibliográficas, como o \textcite{Zotero}\index{Zotero}, o \textcite{Mendeley2009} ou o \textcite{JabRef2009}\index{JabRef}, para a organização das fontes em um arquivo \texttt{.bib}. Essa prática facilita a inserção e atualização de citações no texto por meio dos comandos \verb|\autocite{}|, \verb|\textcite{}| e outros recursos do pacote \textsf{BibLaTeX}. 

A lista de referências deste documento foi gerada automaticamente pelo sistema \LaTeX{} em conjunto com o \textsf{BibLaTeX}, a partir do arquivo {\ttfamily refbase.bib}, elaborado com o auxílio do gerenciador de referências Zotero.

